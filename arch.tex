\documentclass[12pt]{article}
\usepackage[margin=1in]{geometry}
\usepackage{verbatim}
\usepackage{float}
\usepackage{multirow}
\title{\textbf{Arch Install Instructions}}
\date{\today}
\author{Colin Murphy}
\begin{document}

\clearpage\maketitle
\maketitle

\section*{Introduction}
    I am an Electrical and Computer Engineering major at the University of
    Rochester. I have been using Linux for a little over a year and have taken
    a liking to Arch. Although it is more involved to install and configure
    than many distributions, it provides everything you need to make it exactly
    what you want. With that in mind, I have detailed the installation process
    I use for my Arch machines. A lot of these steps are necessary and others
    are more personal preference. I hope you find this helpful and please let
    me know if you have any suggestions for improvement.

\section*{Making bootable}
    If you're running unix system already you can just use the following command
    (THIS WILL REMOVE ALL DATA FROM SDX):
    \begin{verbatim} sudo dd if=/path_to_arch_.iso of=/dev/sdX \end{verbatim}
    Otherwise, use UNetbootin.

\section*{Internet}
    Ping to make sure you are connected to the internet.
    \begin{verbatim} ping -c 3 google.com \end{verbatim}
    If the ping is unsuccessful, try to establish a connection
    \begin{verbatim} dhcpcd \end{verbatim}

\section*{Partitioning}
    Choosing between fdisk (MBR) and gdisk (GPT):
        \begin{itemize}
            \item If using GRUB legacy as the bootloader, one must use MBR.
            \item To dual-boot with Windows (both 32-bit and 64-bit) using
                Legacy BIOS, one must use MBR.
            \item To dual-boot Windows 64-bit using UEFI instead of BIOS, one
                must use GPT.
            \item To dual boot with OS X, it is recommended that you use GPT,
                unless you plan to triple boot with Windows.
            \item If none of the above apply, choose freely between GPT and
                MBR; since GPT is more modern, it is recommended in this case.
            \item It is recommended to use always GPT for UEFI boot as some
                UEFI firmwares do not allow UEFI-MBR boot
        \end{itemize}

\section*{Partition Scheme}
    The partition scheme you use is a matter of personal preference. The
    simplest is:
    \begin{itemize}
        \item Boot partition
        \item Swap partition
        \item Root partition
    \end{itemize}

    \noindent Adding a discrete home partition offers a few notable advantages; This can
    allow you to change distributions or reinstall while keeping your home
    folder intact as well as let you share you home folder among multiple
    installations. On more modern systems, the swap partition is less critical,
    being used for data overflow from RAM. Inclusion of a swap partition is left
    to the discretion of the reader. Typical sizes for each type of partition
    can be seen in Figure \ref{part_sizes}.

    \begin{table}[H]
        \centering
        \caption{Typical Partition Sizes}
        \label{part_sizes}
        \begin{tabular}{|c|c|} \hline
            Boot & 200MB \\ \hline
            Swap & 2GB \\ \hline
            Root & 8GB \\ \hline
            Home & Rest of drive \\ \hline
        \end{tabular}
    \end{table}

\section*{Creating File Systems}
    Now that you have your partitions you need to format them. You can leave
    the boot partition as is but home and root folders need to be ext4. To do
    this, use the following command where sdx is the drive you've partition and
    \# is the partition number:
    \begin{verbatim} mkfs.ext4 /dev/sdx# \end{verbatim}
    For swap, you just need to mark it as swap and turn it on:
    \begin{verbatim} mkswap /dev/sdx# && swapon /dev/sdx# \end{verbatim}

\section*{Mount Partitions}
    Mount you root partition in /mnt:
    \begin{verbatim} mount /dev/sdx# /mnt \end{verbatim}
    then make your home directory and mount your home partition:
    \begin{verbatim} mkdir /mnt/home && mount /dev/sdx# \end{verbatim}

\section*{Installing Base Packages}
    To make downloading the base quicker, modify /etc/pacman.d/mirrorlist with
    nano (the default text editor included with arch) by copying the location
    closest to you at the top. You can search using Ctrl+w, copy using Alt+6
    and paste using Ctrl+u. Then install the base packages:
    \begin{verbatim} pacstrap -i /mnt base base-devel \end{verbatim}

\section*{Configuring File Systems Table}
    Generate an fstab file with the command:
    \begin{verbatim} genfstab -U -p /mnt >> /mnt/etc/fstab \end{verbatim}
    Once you have done this you can chroot into the system to configure it:
    \begin{verbatim} arch-chroot /mnt \end{verbatim}

\section*{Configure Language and Locale}
    Edit /etc/locale.gen and look for your language and uncomment it. Since you're reading this
    in English, you probably want en\_US.UTF-8 UTF-8. To finish configuring
    language and locale, run:
    \begin{verbatim} locale-gen \end{verbatim}
    \begin{verbatim} echo LANG=en_US.UTF-8 > /etc/locale.conf \end{verbatim}
    \begin{verbatim} export LANG=en_US.UTF-8 \end{verbatim}

\section*{Time Zone}
    If you do not know the name of your sub-time zone you can run the following
    to see a list of them:
    \begin{verbatim} ls /usr/share/zoneinfo/ \end{verbatim}
    Next, symbolically link your sub-time zone to localtime and configure the
    hardware clock:
    \begin{verbatim} ln -s /usr/share/zoneinfo/America/New_York> /etc/localtime \end {verbatim}
    \begin{verbatim} hwclock --systohc --utc \end {verbatim}

\section*{Hostname}
    Making your hostname is simple; just run:
    \begin{verbatim} echo your_hostname > /etc/hostname \end{verbatim}

\section*{Configuring Multilib}
    If you are running a 64-bit system you should include the multilib
    repository for pacman. This will add some libraries 32-bit applications may
    need to run on your system. In /etc/pacman.conf, uncomment:
    \begin{verbatim} [multilib] \end{verbatim}
    \begin{verbatim} Include = /etc/pacman.d/mirrorlist \end{verbatim}
    Then run the following to update your package list:
    \begin{verbatim} pacman -Sy \end{verbatim}

\section*{Users}
    You should definitely set the passsword for root which you can do using:
    \begin{verbatim} passwd \end{verbatim}
    Next you can add users:
    \begin{verbatim} useradd -m -g users -G wheel,storage,power -s /bin/bash your_username \end{verbatim}
    Then make the password for a user:
    \begin{verbatim} passwd user_name \end{verbatim}
    Install sudo and bash-completion:
    \begin{verbatim} pacman -S sudo bash-completion \end{verbatim}
    In order to allow users in wheel to use sudo:
    \begin{verbatim} EDITOR=nano visudo \end{verbatim}
    Then uncomment:
    \begin{verbatim} %wheel ALL=(ALL) ALL \end{verbatim}

\section*{Installing GRUB}
    Install grub and os-prober (os-prober is useful is you have multiple OSes):
    \begin{verbatim} pacman -S grub os-prober \end{verbatim}
    \begin{verbatim} grub-install --target=i386-pc --recheck /dev/sdx \end{verbatim}
    If you do have multiple OSes, run the following to update your grub
    configuration:
    \begin{verbatim} grub-mkconfig -o /boot/grub/grub.cfg \end{verbatim}

\section*{Enable Network}
    Find the name of your LAN adapter which will be listed in:
    \begin{verbatim} ip link \end{verbatim}
    Then enable it so it will start on startup:
    \begin{verbatim} systemctl enable dhcpcd@name_of_interface.service
    \end{verbatim}

\section*{Unmount and reboot}
    Now you can exit, unmout and reboot:
    \begin{verbatim} exit \end{verbatim}
    \begin{verbatim} umount -R /mnt \end{verbatim}
    \begin{verbatim} reboot \end{verbatim}
    Depending on your boot order, you may want to remove the USB so you boot
    into your new system rather than the USB.

\section*{Install X}
    Now we need to install X:
    \begin{verbatim} pacman -S xorg-server xorg-server-utils xorg-xinit mesa \end{verbatim}
    Then install the appropriate graphics drivers. This can get hairy and you
    will probably have to do some research to find the correct ones. Running
    the following can help when trying to determine which drivers you need,
    especially if you aren't sure what kind of graphics card you have:
    \begin{verbatim} lspci | grep VGA \end{verbatim}
    A few common devices are listed in Table \ref{graphics_drivers} along with
    the driver associated with them as well as the multilib package.

    \begin{table}[H]
        \centering
        \caption{Common Graphics Devices}
        \label{graphics_drivers}
        \begin{tabular}{|c|c|c|c|} \hline
            Device & Type & Driver & Multilib Driver \\ \hline
            AMD/ATI & Open Source & xf86-video-ati & lib32-ati-dri \\ \cline{2-4}
             & Proprietary & catalyst-dkms & lib32-catalyst-utils \\ \cline{2-4} \hline

            Intel & Open Source & xf86-video-intel & lib32-intel-dri \\ \cline{2-4} \hline

            & Open Source & xf86-video-nouveau & lib32-nouveau-dri \\ \cline{3-4}
                    & & xf86-video-nv & -------- \\ \cline{2-4}

            Nvidia & & nvidia & lib32-nvidia-libgl \\ \cline{3-4}
            & Proprietary & nvidia-304xx & lib32-nvidia-304xx-utils \\ \cline{3-4}
            & & nvidia-173xx & lib32-nvidia-173xx-utils \\ \cline{3-4}
            & & nvidia-96xx & lib32-nvidia-96xx-utils \\ \cline{3-4} \hline
            VIA & Open Source & xf86-video-openchrome & -------- \\ \hline
        \end{tabular}
    \end{table}

    If you are using a laptop you can install the synaptics package that lets
    you use input devices like your track pad:
    \begin{verbatim} pacman -S xf86-input-synaptics \end{verbatim}
    Let's finish installing X and test it to make sure it's working:
    \begin{verbatim} pacman -S xorg-twm xorg-xclock xterm \end{verbatim}
    \begin{verbatim} startx \end{verbatim}
    You should see a few terminal windows an a clock. Type exit to exit.

\section*{Desktop Environment}
    Choosing a desktop environment is completely up to each user to decide
    for themselves; I personally like Cinnamon so I will be using that as an
    example.

    First, install the desktop manager associated with your prefered
    environment. In my case this is gdm. I then install the Cinnamon desktop
    environment.
    \begin{verbatim} sudo pacman -S gdm \end{verbatim}
    \begin{verbatim} sudo pacman -S cinnamon \end{verbatim}
    It's a good idea to test the manager and environment before enabling them
    using systemcl:
    \begin{verbatim} sudo systemctl start gdm \end{verbatim}
    If you have no issues you should see a login screen in your selected dektop
    manager. If that went well, enable it on startup:
    \begin{verbatim} sudo systemctl enable gdm \end{verbatim}

\section*{Network Manager}
    One final thing to configure is your network manager:
    \begin{verbatim} sudo systemctl enable NetworkManager \end{verbatim}
    \begin{verbatim} sudo systemctl start NetworkManager \end{verbatim}
    If you have issues upon rebooting, you can try the following command then
    reboot:
    \begin{verbatim} systemctl disable dhcpcd.service \end{verbatim}

\section*{AUR}
    As a final note I would like to quickly mention how to install packages
    from the Arch Unofficial Repositories which has a huge variety of software
    not available in the official repositories. It is a great resources and
    very well supported.
    As an example I will use google chrome. Once you have found the page on the
    AUR, click "Download tarball". Navigate to where the tarball downloaded
    then run:
    \begin{verbatim} tar -xvf google-chrome.tar.gz \end{verbatim}
    This will create a folder called google-chrome. Go into that folder and
    type:
    \begin{verbatim} makepkg \end{verbatim}
    You may run into dependancies so if you do just install those through
    pacman. Otherwise you should see a file in that folder with a .tar.xz file
    extension. To install the pacman, just type:
    \begin{verbatim} sudo pacman -U google-chrome-39.0.2171.95-1-x86_64.pkg.tar.xz \end{verbatim}

  \section*{Yaourt}
    You can also use a wrapper such as yaourt to handle the installation of AUR
    packages. Wrappers such as this handle dependancies.
    To install Yaourt you must first install package-query. This is done using
    the meathod described above.

    \begin{verbatim}curl -O https://aur.archlinux.org/packages/pa/package-query/package-query.tar.gz\end{verbatim}
    \begin{verbatim}tar zxvf package-query.tar.gz\end{verbatim}
    \begin{verbatim}cd package-query\end{verbatim}
    \begin{verbatim}makepkg -si\end{verbatim}

    Then install yaourt
    \begin{verbatim}curl -O https://aur.archlinux.org/packages/ya/yaourt/yaourt.tar.gz\end{verbatim}
    \begin{verbatim}tar zxvf yaourt.tar.gz\end{verbatim}
    \begin{verbatim}cd yaourt\end{verbatim}
    \begin{verbatim}makepkg -si \end{verbatim}

    To use yaourt simply type
    \begin{verbatim}yaourt <package name>\end{verbatim}
    Otherwise the syntax is the same as pacman.


\end{document}
